\documentclass[11pt]{article}
\usepackage[a4paper,margin=1.5cm]{geometry}
\usepackage{mathtools,amsmath,nccmath,amssymb}

\setlength{\parindent}{0pt}
\renewcommand{\arraystretch}{1.2}

\newenvironment{question}[1]
  {\par\addvspace{\medskipamount}
   \noindent\makebox[0pt][r]{\textbf{#1)} }\ignorespaces}
  {\par\addvspace{\medskipamount}}


\begin{document}
\begin{question}{1}
$A=\{(4, 2, 0),(1, -1, 1),(5, 3, 3)\}$

$B=\{(1, -2, 1),(1, 5, 2),(1, 0, 1)\}$

\bigskip
\textbf{Fórmulas genéricas para a base A}

\bigskip
\textit{Gerando o sistema:} 

$$a_1(4,2,0)+a_2(1,-1,1)+a_3(5,3,3)=(x,y,z)$$

$$
\left\{\begin{array}{rcrcrcr}
    4a_1 &+ &a_2 &+ &5a_3 &= &x\\
    2a_1 &- &a_2 &+ &3a_3 &= &y\\
         &  &a_2 &+ &3a_3 &= &z
\end{array}\right.
$$

\textit{Eq. 3:}
$$
\begin{array}{rcl}
    a_2+3a_3 &= &z\\
    a_2 &= &-3a_3+z
\end{array}
$$

\bigskip
\textit{Substituindo Eq. 2:}
$$
\begin{array}{rcl}
    2a_1-a_2+3a_3 &= &y\\
    2a_1+3a_3-z+3a_3 &= &y\\
    2a_1 &= &-6a_3+y+z
\end{array}
$$

\bigskip
\textit{Substituindo Eq. 1:}
$$
\begin{array}{rcl}
    4a_1+a_2+5a_3 &= &x\\
    2(-6a_3+y+z)-3a_3+z+5a_3 &= &x\\
    -10a_3 &= &x-2y-3z\\
    a_3 &= &\frac{1}{10}(-x+2y+3z)
\end{array}
$$

\bigskip
\textit{Substituindo Eq. 3:}
$$
\begin{array}{rcl}
    a_2 &= &-3a_3+z\\
    a_2 &= &-\frac{3}{10}(-x+2y+3z)+z\\
    a_2 &= &\frac{1}{10}(3x-6y+z)
\end{array}
$$

\bigskip
\textit{Substituindo Eq. 2:}
$$
\begin{array}{rcl}
    2a_1 &= &-6a_3+y+z\\
    2a_1 &= &-\frac{6}{10}(-x+2y+3z)+y+z\\
    a_1 &= &\frac{1}{10}(3x-y-4z)
\end{array}
$$

\bigskip
\textit{Formulas da base A:}
$$
\begin{array}{rcl}
    a_1(x,y,z) &= &\frac{1}{10}(3x-y-4z)\\
    a_2(x,y,z) &= &\frac{1}{10}(3x-6y+z)\\
    a_3(x,y,z) &= &\frac{1}{10}(-x+2y+3z)
\end{array}
$$

\newpage
\bigskip
\textbf{Fórmulas genéricas para a base B}

\bigskip
\textit{Gerando o sistema:} 

$$b_1(1,-2,1)+b_2(1,5,2)+b_3(1,0,1)=(x,y,z)$$

$$
\left\{\begin{array}{rcrcrcr}
    b_1   &+ &b_2  &+ &b_3 &= &x\\
    -2b_1 &+ &5b_2 &  &    &= &y\\
    b_1   &+ &2b_2 &+ &b_3 &= &z
\end{array}\right.
$$

\textit{Eq. 2:}
$$
\begin{array}{rcl}
    -2b_1 + 5b_2 &= &y\\
    b_1 &= &\frac{5}{2}b_2-\frac{y}{2}
\end{array}
$$

\bigskip
\textit{Substituindo Eq. 3:}
$$
\begin{array}{rcl}
    b_1+2b_2+b_3 &= &z\\
    \frac{5}{2}b_2-\frac{y}{2}+2b_2+b_3 &= &z\\
    b_3 &= &-\frac{9}{2}b_2+\frac{y}{2}+z
\end{array}
$$

\bigskip
\textit{Substituindo Eq. 1:}
$$
\begin{array}{rcl}
    b_1+b_2+b_3 &= &x\\
    \frac{5}{2}b_2-\frac{y}{2}+b_2+-\frac{9}{2}b_2+\frac{y}{2}+z &= &x\\
    b_2 &= &-x+z
\end{array}
$$

\bigskip
\textit{Substituindo Eq. 2:}
$$
\begin{array}{rcl}
    b_1 &= &\frac{5}{2}b_2-\frac{y}{2}\\
    b_1 &= &\frac{5}{2}(-x+z)-\frac{y}{2}\\
    b_1 &= &\frac{1}{2}(-5x-y+5z)
\end{array}
$$

\bigskip
\textit{Substituindo Eq. 3:}
$$
\begin{array}{rcl}
    b_3 &= &-\frac{9}{2}b_2+\frac{y}{2}+z\\
    b_3 &= &-\frac{9}{2}(-x+z)+\frac{y}{2}+z\\
    b_3 &= &\frac{1}{2}(9x+y-7z)
\end{array}
$$

\bigskip
\textit{Formulas da base B:}
$$
\begin{array}{rcl}
    b_1(x,y,z) &= &\frac{1}{2}(-5x-y+5z)\\
    b_2(x,y,z) &= &-x+z\\
    b_3(x,y,z) &= &\frac{1}{2}(9x+y-7z)
\end{array}
$$

\newpage
\bigskip
\textbf{$\boldsymbol{M_{B \rightarrow A}}$:}

$$b_1 = x_1 \cdot a_1+y_1 \cdot a_2+z_1 \cdot a_3=(1,-2,1)$$

$$
\begin{array}{rcrcl}
    x_1 &= &a_1(1,-2,1) &= &\frac{1}{10}\\
    y_1 &= &a_2(1,-2,1) &= &\frac{8}{5}\\
    z_1 &= &a_3(1,-2,1) &= &-\frac{1}{5}
\end{array}
$$

$$b_2 = x_2 \cdot a_1+y_2 \cdot a_2+z_2 \cdot a_3=(1,5,2)$$

$$
\begin{array}{rcrcl}
    x_2 &= &a_1(1,5,2) &= &-1\\
    y_2 &= &a_2(1,5,2) &= &-\frac{5}{2}\\
    z_2 &= &a_3(1,5,2) &= &\frac{3}{2}
\end{array}
$$

$$b_3 = x_3 \cdot a_1+y_3 \cdot a_2+z_3 \cdot a_3=(1,0,1)$$

$$
\begin{array}{rcrcl}
    x_3 &= &a_1(1,0,1) &= &-\frac{1}{10}\\
    y_3 &= &a_2(1,0,1) &= &\frac{2}{5}\\
    z_3 &= &a_3(1,0,1) &= &\frac{1}{5}
\end{array}
$$

$$
M_{B \rightarrow A}=\left[\begin{array}{rrr}
    \frac{1}{10} &-1 &-\frac{1}{10}\\
    \frac{8}{5} &-\frac{5}{2} &\frac{2}{5}\\
    -\frac{1}{5} &\frac{3}{2} &\frac{1}{5}
\end{array}\right]
$$

\bigskip
\textbf{$\boldsymbol{M_{A \rightarrow B}}$:}

$$b_1 = x_1 \cdot b_1+y_1 \cdot b_2+z_1 \cdot b_3=(4,2,0)$$

$$
\begin{array}{rcrcl}
    x_1 &= &b_1(4,2,0) &= &-11\\
    y_1 &= &b_2(4,2,0) &= &-4\\
    z_1 &= &b_3(4,2,0) &= &19
\end{array}
$$

$$b_2 = x_2 \cdot b_1+y_2 \cdot b_2+z_2 \cdot b_3=(1,-1,1)$$

$$
\begin{array}{rcrcl}
    x_2 &= &b_1(1,-1,1) &= &\frac{1}{2}\\
    y_2 &= &b_2(1,-1,1) &= &0\\
    z_2 &= &b_3(1,-1,1) &= &\frac{1}{2}
\end{array}
$$

$$b_3 = x_3 \cdot b_1+y_3 \cdot b_2+z_3 \cdot b_3=(5,3,3)$$

$$
\begin{array}{rcrcl}
    x_3 &= &b_1(5,3,3) &= &-\frac{13}{2}\\
    y_3 &= &b_2(5,3,3) &= &2\\
    z_3 &= &b_3(5,3,3) &= &\frac{27}{2}
\end{array}
$$

$$
M_{A \rightarrow B}=\left[\begin{array}{rrr}
    -11 &\frac{1}{2} &-\frac{13}{2}\\
    -4 &0 &2\\
    19 &\frac{1}{2} &\frac{27}{2}
\end{array}\right]
$$

\newpage
\textbf{a)} $v=(0,1,2)_A$ em $B$:

$$
M_{A \rightarrow B}:
\left[\begin{array}{rrr}
    -11 &\frac{1}{2} &-\frac{13}{2}\\
    -4 &0 &2\\
    19 &\frac{1}{2} &\frac{27}{2}
\end{array}\right]
\cdot
\left[\begin{array}{r}
    0\\
    1\\
    2\\
\end{array}\right]
=
\left[\begin{array}{r}
    -\frac{25}{2}\\
    -4\\
    \frac{55}{2}\\
\end{array}\right]
$$

$$v=(-\frac{25}{2},-4,\frac{55}{2})_B$$

\textbf{b)} $v=(1,3,-1)_B$ em $A$:

$$
M_{b \rightarrow a}:
\left[\begin{array}{rrr}
    \frac{1}{10} &-1 &-\frac{1}{10}\\
    \frac{8}{5} &-\frac{5}{2} &\frac{2}{5}\\
    -\frac{1}{5} &\frac{3}{2} &\frac{1}{5}
\end{array}\right]
\cdot
\left[\begin{array}{r}
    1\\
    3\\
    -1\\
\end{array}\right]
=
\left[\begin{array}{r}
    -\frac{14}{5}\\
    -\frac{63}{10}\\
    \frac{41}{10}\\
\end{array}\right]
$$

$$v=(-\frac{14}{5},-\frac{63}{10},\frac{41}{10})_A$$

\end{question}

\begin{question}{2}
$S = \{(1, 1, 1, 1, 1),(2, 0, -1, 1, 3),(3, 1, 0, 2, 4),(2, 2, 5, 8, -1),(0, 1, 0, 2, 3)\}$

\bigskip
\textbf{a)} $S$ é li ou ld?

\bigskip
\textit{Gerando o sistema:} 

$$x_1(1,1,1,1,1)+x_2(2,0,-1,1,3)+x_3(3,1,0,2,4)+x_4(2,2,5,8,-1)+x_5(0,1,0,2,3)=(0,0,0,0,0)$$

$$
\left\{\begin{array}{rcrcrcrcrcl}
    x_1 &+ &2x_2 &+ &3x_3 &+ &2x_4 &  &     &= &0\\
    x_1 &  &     &+ &x_3  &+ &2x_4 &+ &x_5  &= &0\\
    x_1 &- &x_2  &  &     &+ &5x_4 &  &     &= &0\\
    x_1 &+ &x_2  &+ &2x_3 &+ &8x_4 &+ &2x_5 &= &0\\
    x_1 &+ &3x_2 &+ &4x_3 &- &x_4  &+ &3x_5 &= &0\\
\end{array}\right.
$$

\textit{Escalonamento:}

$$
\left[\begin{array}{rrrrr|c}
  1 & 2  & 3 & 2  & 0 & 0\\
  1 & 0  & 1 & 2  & 1 & 0\\
  1 & -1 & 0 & 5  & 0 & 0\\
  1 & 1  & 2 & 8  & 2 & 0\\
  1 & 3  & 4 & -1 & 3 & 0
\end{array}\right]
\substack{
  \mbox{$L_2=L_2-L_1 \rightarrow$}\\[.5em]
  \mbox{$L_3=L_3-L_1 \rightarrow$}\\[.5em]
  \mbox{$L_4=L_4-L_1 \rightarrow$}\\[.5em]
  \mbox{$L_5=L_5-L_1 \rightarrow$}
}
\left[\begin{array}{rrrrr|c}
  1 & 2  & 3  & 2  & 0 & 0\\
  0 & -2 & -2 & 0  & 1 & 0\\
  0 & -3 & -3 & 3  & 0 & 0\\
  0 & -1 & -1 & 6  & 2 & 0\\
  0 & 1  & 1  & -3 & 3 & 0
\end{array}\right]
$$

$$
\substack{
  \mbox{$L_3=L_3-\mfrac{3}{2}L_2 \rightarrow$}\\[.5em]
  \mbox{$L_4=L_4-\mfrac{L_2}{2} \rightarrow$}\\[.5em]
  \mbox{$L_5=L_5+\mfrac{L_2}{2} \rightarrow$}
}
\left[\begin{array}{rrrrr|c}
  1 & 2  & 3  & 2  & 0 & 0\\
  0 & -2 & -2 & 0  & 1 & 0\\
  0 & 0  & 0  & 3  & -\frac{3}{2} & 0\\
  0 & 0  & 0  & 6  & \frac{3}{2} & 0\\
  0 & 0  & 0  & -3 & \frac{7}{2} & 0
\end{array}\right]
\substack{
  \mbox{$L_4=L_4-2L_3 \rightarrow$}\\[.5em]
  \mbox{$L_5=L_5-L_3 \rightarrow$}
}
\left[\begin{array}{rrrrr|c}
  1 & 2  & 3  & 2  & 0 & 0\\
  0 & -2 & -2 & 0  & 1 & 0\\
  0 & 0  & 0  & 3  & -\frac{3}{2} & 0\\
  0 & 0  & 0  & 0  & 2 & 0\\
  0 & 0  & 0  & 0 & 5 & 0
\end{array}\right]
$$

\textit{Substituição:}

$$
\left\{\begin{array}{rcrcrcrcrcr}
  x_1 &+ &2x_2  &+ &3x_3 &+ &2x_4 &  &     &= &0\\
      &  &-2x_2 &- &2x_3 &  &     &+ &x_5  &= &0\\
      &  &      &  &     &  &3x_4 &- &\frac{3}{2}x_5 &= &0\\
      &  &      &  &     &  &     &  &2x_5 &= &0\\
      &  &      &  &     &  &     &  &5x_5 &= &0\\
\end{array}\right.
$$

\textit{Eq. 4:}
$$
\begin{array}{rcl}
  2x_5 &= &0\\
  x_5 &= &0\\
\end{array}
$$

\textit{Eq. 3:}
$$
\begin{array}{rcl}
  3x_4 &= &0\\
  x_4 &= &0\\
\end{array}
$$

\textit{Substituindo Eq. 2:}
$$
\begin{array}{rcl}
  -2x_2-2x_3+x_5  &= &0\\
  -2x_2-2x_3+0  &= &0\\
  x_2 = -x_3
\end{array}
$$

\textit{Substituindo Eq. 1:}
$$
\begin{array}{rcl}
  x_1+2x_2+3x_3+2x_4 &= &0\\
  x_1-2x_3+3x_3+0 &= &0\\
  x_1 = -x_3
\end{array}
$$

\textit{Resultados:}
$$
\begin{array}{rcl}
  x_1 &= &-x_3\\
  x_2 &= &-x_3\\
  x_3 &= &?\\
  x_4 &= &0\\
  x_5 &= &0\\
\end{array}
$$

\textit{Resposta:} O conjunto $S$ é LD, pois $x_3$ pode ter qualquer valor para encontrar $(0,0,0,0,0)$. Por exemplo:

$$-1(1,1,1,1,1)+(-1)(2,0,-1,1,3)+1(3,1,0,2,4)+0(2,2,5,8,-1)+0(0,1,0,2,3)=(0,0,0,0,0)$$

\textbf{b)} $S$ forma uma base do $\mathbb{R}$-espaço vetorial $\mathbb{R}^5$?

\textit{Resposta:} Não, pois apenas conjuntos LI podem ser bases de espaços vetoriais.


\end{question}

\end{document}